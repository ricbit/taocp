\documentclass[10pt]{book}
\usepackage[utf8]{inputenc}
\usepackage{verbatim}
\usepackage{amsmath}
\usepackage{amsthm}
\usepackage{times}

\begin{document}
\title{Fascicle 6 Satisfiability, exercises 297-299}
\author{Ricardo Bittencourt}
\maketitle

\chapter{4.2.2.2 Satisfiability}

\section{Exercise 297 item a}

Find a closed form for $G_q(z)=\sum_p C_{p,p+q-1}(z/3)^{p+q}(2z/3)^p$, using formulas from Section 7.2.1.6.

\begin{proof}
Let's take $(z/3)^q$ outside:

\begin{align*}
  \sum_p C_{p,p+q-1}\left(\frac{z}{3}\right)^{p+q}\left(\frac{2z}{3}\right)^p 
  &= \left(\frac{z}{3}\right)^q\sum_p C_{p,p+q-1}\left(\frac{z}{3}\right)^p\left(\frac{2z}{3}\right)^p \\
  &= \left(\frac{z}{3}\right)^q\sum_p C_{p,p+q-1}\left(\frac{2z^2}{9}\right)^p
\end{align*}

If we make $y=2z^2/9$, then we can use eq. 24 from the book:

$$C_{pq}=[z^p]C(z)^{q-p+1} \implies \sum_pC_{pq}z^p = C(z)^{q-p+1}$$

Which in our case is:

\begin{align*}
  \left(\frac{z}{3}\right)^q\sum_p C_{p,p+q-1}\left(\frac{2z^2}{9}\right)^p
  &= \left(\frac{z}{3}\right)^q C\left(\frac{2z^2}{9}\right)^{(p+q-1)-p+1} \\
  &= \left(\frac{z}{3}\right)^q C\left(\frac{2z^2}{9}\right)^q \\
\end{align*}

Let's check eq. 18 from the book:

$$C(z)=\frac{1-\sqrt{1-4z}}{2z}$$

In our case:

\begin{align*}
  \left(\frac{z}{3}\right)^q C\left(\frac{2z^2}{9}\right)^q 
    &= \left(\frac{z}{3}\right)^q 
       \left( \frac{1-\sqrt{1-4(2z^2/9)}}{2(2z^2/9)} \right) ^q \\
       &= \left(\frac{3}{4z}\right)^q
       \left(1-\sqrt{1-\frac{8z^2}{9}}\right)^q \\
  &= \left(\frac{3-\sqrt{9-8z^2}}{4z}\right)^q
\end{align*}

\qedhere
\end{proof}

\section{Exercise 297 item b}

Explain why $G_q(1)$ is less than 1.

\begin{proof}
$$G_q(1)=\left(\frac{3-\sqrt{9-8}}{4}\right)^q=2^{-q}$$

This is indeed less than 1 for all $q>0$. Let's interpret this result. $q$ is a measure of how "distant" the initial configuration is from the solution. $t=2p+q$ is the time taken to reach a solution when the path rises by $p$. 

$C_{p,p+q-1}$ is the number of possible paths from $q$ to the solution of size $t$. Each of these paths has a chance of $(1/3)^{p+q}(2/3)^q$ to happen. Therefore, if you sum over all possible $p$, then it's the same as summing over all possible $t$, and the meaning of $G_q(1)$ is the chance of reaching the solution when starting from $q$, with any possible time. Since it is a chance, it must be smaller than 1.

\end{proof}

\section{Exercise 297 item c}

Evaluate and interpret the quantity $G'_q(1)/G_q(1)$.

\begin{proof}
Let's use Bayes' Rule:

$$P(\text{finishes at} \,t \mid \text{finishes at all}) =
\frac{P(\text{finishes at all} \mid \text{finishes at} \,t)
P(\text{finishes at} \,t)}{P(\text{finishes at all})}$$

      P(finishes at all $\mid$ finishes at t) is clearly $1$. P(finishes at t) is $[z^t]G_q(z)$. P(finishes at all) is $G_q(1)$.

      Therefore:

\begin{align*}
  \frac{G_q(z)}{G_q(1)} &= 
  \sum P(\text{finishes at }\,t\mid\text{finishes at all}) z^t\\
  \frac{G'_q(z)}{G_q(1)} &= 
  \sum t P(\text{finishes at }\,t\mid\text{finishes at all}) z^{t-1}\\
  \frac{G'_q(1)}{G_q(1)} &= 
  E(t) \mid\text{finishes at all}
\end{align*}

The interpretation for $G'_q(1)/G_q(1)$ is the expected time to reach the solution, given it has finished at all.

After long calculation. $G'_q(z)$ turns out to be:
$$G'_q(z)=\frac{3q}{z\sqrt{9-8z^2}}\left(\frac{3-\sqrt{9-8z^2}}{4z}\right)^q$$

And therefore $G'_q(1)/G_q(1)=3q$.

We can also calculate the variance as:

$$\left(\frac{G''_q(1)+G'_q(1)}{G_q(1)}\right)-\left(\frac{G'_q(1)}{G_q(1)}\right)^2=24q$$

(This last calculation made by Mathematica).

\end{proof}

\section{Exercise 297 item d}

Use Markov's inequality to bound the probability that $Y_t=0$ for some $t\le N$.

\begin{proof}
  The probability that a given state is distant $q$ bits from the solution is the number of $n$-sized binary words with exactly $q$ bits set (this is ${n\choose q}$), divided by the total number of $n$-sized binary words, which is $2^n$. The generating function for the probability of the algorithm ending when running with $n$ variables is:

$$ T(z)=\sum_q {n\choose q}2^{-n}G_q(z) $$

Using the same argument as the last exercises, the expected path length to the solution, given that it terminates, is $E[t] = T'(1)/T(1)$:

\begin{align*}
  T(1) &= \sum_q {n\choose q}2^{-n}G_q(1) \\
       &= 2^{-n} \sum_q {n\choose q}2^{-q} \\
       &= 2^{-n} \sum_q {n\choose q}\left(\frac{1}{2}\right)^{q}\left(1\right)^{n-q} \\
    &= 2^{-n}\left(\frac{1}{2}+1\right)^n \\
    &= \left(\frac{3}{4}\right)^n
\end{align*}

\begin{align*}
  T'(1) &= \sum_q {n\choose q}2^{-n}G'_q(1) \\
  &= 2^{-n} \sum_q {n\choose q}3q 2^{-q} \\
  &= 3\times 2^{-n} \sum_q q{n\choose q}\left(\frac{1}{2}\right)^{q} \\
  &= 3\times 2^{-n} \sum_q n{n-1\choose q-1}\left(\frac{1}{2}\right)^{q} \\
  &= 3n2^{-n} \sum_{q\ge 0} {n-1\choose q-1}\left(\frac{1}{2}\right)^{q} \\
  &= \frac{3n2^{-n}}{2} \sum_{q\ge 0} {n-1\choose q-1}\left(\frac{1}{2}\right)^{q-1} \\
  &= \frac{3n2^{-n}}{2} \left({n-1\choose -1}\left(\frac{1}{2}\right)^{-1}+
\sum_{q-1\ge 0} {n-1\choose q-1}\left(\frac{1}{2}\right)^{q-1}\right) \\
  &= \frac{3}{2}n2^{-n}\left(\frac{3}{2}\right)^{n-1}  \\
  &= n2^{-n}\left(\frac{3}{2}\right)^{n}  \\
  &= n\left(\frac{3}{4}\right)^{n}  \\
\end{align*}

$$  \frac{T'(1)}{T(1)} = \frac{n\left(\frac{3}{4}\right)^n}
  {\left(\frac{3}{4}\right)^n} 
  = n$$

Now we can use Markov's inequality:

$$P(X\ge a) \le \frac{E[X]}{a}$$

Which in our case is:

$$P(X\ge N|\text{given the algorith terminates}) \le \frac{n}{N}$$

\end{proof}

\section{Exercise 297 item e}

Show that Corollary W follows from this analysis.

\begin{proof}

Let's define $p$ as the probability that the algorithm succeeds within $N$ steps. Then:

\begin{align*}
  P(\text{succeeds within N}) &= P(\text{succeeds within N}\,\mid\text{succeeds at all})
  P(\text{succeeds at all}) \\
  &=\left(1 - P(X\ge N|\text{succeeds at all})\right)\left(\frac{3}{4}\right)^n \\
  &\ge\left(1 - \frac{n}{N}\right)\left(\frac{3}{4}\right)^n \\
\end{align*}

The probability it succeeds in $Q$ trials is:

\begin{align*}
  P(\text{succeeds in Q trials}) 
    &= p + (1-p)p +(1-p)^2p + \cdots + (1-p)^{Q-1}p \\  
    &= p\frac{1 - (1-p)^Q}{1- (1-p)}\\
    &= 1-(1-p)^Q\\
  &= 1-\exp\left(Q\log\left(1-p\right)\right)\\
  &= 1-\exp\left(-Q\sum p+\frac{p^2}{2}+\frac{p^3}{3}+\cdots\right)\\
  &\ge 1-\exp\left(-Qp\right)
\end{align*}

For $Q=K(4/3)^n$ and $p$ given above, we have:

\begin{align*}
  P(\text{succeeds in Q trials})
  &\ge 1-\exp\left(-K\left(\frac{4}{3}\right)^n \left(1-\frac{n}{N}\right)
\left(\frac{3}{4}\right)^n\right) \\
  &\ge 1-\exp\left(-K\left(1-\frac{n}{N}\right)\right)
\end{align*}

Now we choose $N=2n$ to conclude:

\begin{align*}
  P(\text{succeeds in Q trials})
  &\ge 1-\exp\left(-K\left(1-\frac{n}{2n}\right)\right) \\
  &\ge1-\exp\left(-\frac{K}{2}\right)
\end{align*}

\end{proof}

\section{Exercise 298}

Generalize Theorem U and Corollary W to the case where each clause has at most $k$ clauses, where $k\le 3$.

\begin{proof}
  We'll follow the steps of the previous exercise, but swapping the probabilities $1/3$ and $2/3$ to $1/k$ and $(k-1)/k$. First, $G_q(z)$ becomes:

  \begin{align*}
    G_q(z) &= \sum_p C_{p,p+q-1}\left(\frac{z}{k}\right)^{p+q}
      \left(\frac{z(k-1)}{k}\right)^p \\
      &= \left(\frac{z}{k}\right)^q 
      \sum_p C_{p,p+q-1}\left(\frac{z^2(k-1)}{k^2}\right)^p \\
      &= \left(\frac{z}{k}\right)^q C\left(\frac{z^2(k-1)}{k^2}\right)^q \\
      &= \left( 
      \frac{z}{k}\left(\frac{1-\sqrt{1-4\frac{z^2(k-1)}{k^2}}}
           {2\frac{z^2(k-1)}{k^2}} \right)
         \right)^q \\
      &= \left(\frac{1}{2z(k-1)}\left(k-\sqrt{k^2-4z^2(k-1)}\right)\right)^q
  \end{align*}

  \begin{align*}
    G_q(1) &= \left( \frac{1}{2(k-1)}\left(k-\sqrt{k^2-4k+4} \right)\right)^q\\
           &= \left( \frac{1}{2(k-1)}\left(k-(k-2) \right)\right)^q\\
           &= \left( \frac{1}{k-1}\right)^q
  \end{align*}

  For the derivative, let's write $G_q(z)$ as $\left(G(z)\right)^q$:

  \begin{align*}
    G(z) 
      &= \frac{1}{2z(k-1)}\left(k-\sqrt{k^2-4z^2(k-1)}\right) \\
    G'(z) 
      &= \frac{k}{2(k-1)z^2}\left(-1+\frac{k}{\sqrt{k^2-4(k-1)z^2}}\right) \\
    G'(1) 
      &= \frac{k}{2(k-1)}\left(-1+\frac{k}{k-2}\right)\\
      &= \frac{k}{(k-1)(k-2)} \\
    G_q'(z) &= qG(z)^{q-1}G'(z) \\
    G_q'(1) &= q\left(\frac{1}{k-1}\right)^{q-1}\frac{k}{(k-1)(k-2)} \\
      &= \left(\frac{1}{k-1}\right)^q \frac{qk}{k-2}
  \end{align*}

  Let's write $T(z)$ in terms of $G(z)$:

  \begin{align*}
    T(z) &= \sum_q {n\choose q}2^{-n}\left(G(z)\right)^q \\
         &= 2^{-n}\left(1+G(z)\right)^n \\
    T'(z) &= n2^{-n} \left(1+G(z)\right)^{n-1}G'(z) \\
    \frac{T'(z)}{T(z)} 
      &= \frac{n2^{-n} \left(1+G(z)\right)^{n-1}G'(z)}
         {2^{-n} \left(1+G(x)\right)^n} \\
      &= \frac{nG'(z)}{1+G(z)} \\
    \frac{T'(1)}{T(1)} 
    &= \frac{\frac{nk}{(k-1)(k-2)}}{1+\frac{1}{k-1}}
     =\frac{nk}{(k-1)(k-2)}\frac{k-1}{k}
     =\frac{n}{k-2}
  \end{align*}

  The probability of success within $N$ steps is:

  \begin{align*}
    p &\ge \left(1-\frac{T'(1)/T(1)}{N}\right)T(1) \\
      &\ge \left(1-\frac{n}{N(k-2)}\right)\left(2^{-n}\left(1+\frac{1}{k-1}\right)^n\right) \\
      &\ge \left(1-\frac{n}{N(k-2)}\right)\left(\frac{k}{2(k-1)}\right)^n
  \end{align*}

  We know that the probability of success with $Q$ trials is $P\ge 1-exp(-Qp)$, so we want to choose $Q$ and $N$ such that $Qp=K/2$. For example, $Q=K(2-2/k)^n$ and $N=2n/(k-2)$:

  \begin{align*}
    Qp &= K\left(2-\frac{2}{k}\right)^n\left(1-\frac{1}{2}\right)
      \left(\frac{k}{2(k-1)}\right)^n\\
      &= \frac{K}{2}
  \end{align*}

  Both $Q$ and $N$ need to be integers. For $Q$, you can always choose a suitable $K$ to make the quantity an integer, but that's not possible for $N$ since $n$ and $k$ are given. Therefore, you must round down $N$ to:

  $$N=\left\lfloor\frac{2n}{k-2}\right\rfloor$$

\end{proof}
\section{Exercise 299}

Continuing the previous exercise, investigate the case $k=2$.

\begin{proof}

  The approach used in previous exercises doesn't work here, because $G'(1)$ will have a division by zero. We'll resort to the original probabilities. If we choose $N=n^2$, then the chance of failure is:

\begin{align*}
  P(X > n^2) &= \sum_{p,q} f(p,q) [2p+q > n^2] \\
      &= \sum_{p,q} \frac{1}{2^n}{n \choose q}\frac{q}{2p+q}
       {2p+q\choose p}
       \left(\frac{1}{2}\right)^{p+q}\left(\frac{1}{2}\right)^{p} [2p+q> n^2]
\end{align*}

Let's make $t=2p+q$, then:

\begin{align*}
  P(X > n^2)
    &= \sum_{p,t} \frac{1}{2^n}{n \choose t-2p}\frac{t-2p}{t}
       {t\choose p}
       \left(\frac{1}{2}\right)^{t} [t> n^2] \\
    &= \frac{1}{2^n}\sum_{t>n^2} \frac{1}{t2^t}\sum_p(t-2p){n \choose t-2p}
       {t\choose p}
\end{align*}

Now we can use the following bound, which is true because the middle element in any row of Pascal's triangle is always the greatest:

$${n\choose p}\le{n\choose\lfloor n/2\rfloor}$$

In our case:

\begin{align*}
  P(X > n^2) 
  &= \frac{1}{2^n}\sum_{t>n^2} \frac{1}{t2^t}\sum_p(t-2p){n \choose t-2p}
       {t\choose p} \\
  &\le \frac{1}{2^n}\sum_{t>n^2} \frac{1}{t2^t}{t\choose\lfloor t/2\rfloor}
       \sum_p(t-2p){n \choose t-2p}\\
  &\le \frac{1}{2^n}\sum_{t>n^2} \frac{1}{t2^t}{t\choose\lfloor t/2\rfloor}
       \sum_q(t-q){n \choose t-q}[q\text{ even}]\\
  &\le \frac{1}{2^n}\sum_{t>n^2} \frac{1}{t2^t}{t\choose\lfloor t/2\rfloor}
       \sum_q(t-q){n \choose t-q}
       \left(\frac{1}{2}\left(1-(-1)^q\right)\right)\\
  &\le \frac{1}{2^n}\sum_{t>n^2} \frac{1}{t2^t}{t\choose\lfloor t/2\rfloor}
       \left(
         \sum_q\frac{t-q}{2}{n \choose t-q} + 
         \sum_q\frac{t-q}{2}{n \choose t-q}(-1)^q 
       \right)
\end{align*}

The second inner sum is zero, because it's a special case of eq. (5.42) from Concrete Mathematics. Namely, the sum of an alternating binomial times a polynomial is zero when the degree of the polynomial is less than n.

\begin{align*}
  P(X > n^2) 
  &\le \frac{1}{2^n}\sum_{t>n^2} \frac{1}{t2^t}{t\choose\lfloor t/2\rfloor}
       \left(
       \frac{1}{2}\sum_q(t-q){n \choose t-q}
       \right) \\
  &\le \frac{1}{2^n}\sum_{t>n^2} \frac{1}{t2^t}{t\choose\lfloor t/2\rfloor}
       \left(
         \frac{1}{2}\sum_k k{n \choose k}
       \right)\\
  &\le \frac{1}{2^n}\sum_{t>n^2} \frac{1}{t2^t}{t\choose\lfloor t/2\rfloor}
       \left(
         \frac{n2^{n-1}}{2}
       \right)\\
  &\le \frac{n}{4}\sum_{t>n^2} \frac{1}{t2^t}{t\choose\lfloor t/2\rfloor}
\end{align*}

From Stirling's approximation we have:

\begin{align*}
  {2n\choose n}&\le \frac{4^n}{\sqrt{\pi n}} \\
  {t\choose \lfloor t/2\rfloor}&\le \frac{2^{t}}{\sqrt{\pi t/2}} \\
\end{align*}

Applied to our case:

\begin{align*}
  P(X > n^2) 
  &\le \frac{n}{4}\sum_{t>n^2} \frac{1}{t2^t}\left(\frac{2^t\sqrt{2}}{\sqrt{\pi t}}\right)\\
  &\le \frac{n}{4}\sum_{t>n^2} \frac{1}{t2^t}\left(\frac{2^t\sqrt{2}}{\sqrt{\pi t}}\right)\\
  &\le \frac{n}{\sqrt{8\pi}}\sum_{t>n^2} \sqrt{\frac{1}{t^3}}\\
\end{align*}

Now we can use the following property:

$$\sum_{x>a}f(x)=\int_a^\infty f(\lceil x\rceil)dx$$

This is just a different way of writing the same thing. We are replacing a sum of values by a sum of rectangles with width $1$ and height $f(x)$, both evaluate numerically to the same value.

\begin{align*}
  P(X > n^2) 
  &\le \frac{n}{\sqrt{8\pi}}\int_{n^2}^\infty \frac{dx}{\lceil x\rceil^{3/2}}
\end{align*}

Since $x^{-3/2}$ is decreasing in this region, we can drop the ceiling:

$$\int_{n^2}^\infty\frac{dx}{\lceil x\rceil^{3/2}} \le 
  \int_{n^2}^\infty\frac{dx}{x^{3/2}} $$

Therefore:

\begin{align*}
  P(X > n^2) 
  &\le \frac{n}{\sqrt{8\pi}}\int_{n^2}^\infty \frac{dx}{ x^{3/2}} \\
  &\le \frac{n}{\sqrt{8\pi}}\left(\left.-\frac{2}{\sqrt{x}}
       \right\rvert^\infty_{n^2}\right) \\
  &\le \frac{1}{\sqrt{2\pi}}
\end{align*}





\end{proof}

\end{document}

